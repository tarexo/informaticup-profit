\documentclass[12pt,a4paper]{article}
\usepackage{array}
\usepackage{color}
\usepackage[ngerman]{babel}
\usepackage{graphicx}
\usepackage{textcomp}
\usepackage{anysize}
\usepackage[
colorlinks=true,
urlcolor=blue,
linkcolor=black,
citecolor = black
]{hyperref}

\usepackage[a4paper,margin=3cm, headsep=24pt, headheight=2cm]{geometry}
\usepackage[square,sort,comma,numbers]{natbib}

\usepackage{fancyhdr}
\pagestyle{fancy} %vordefinierter Style
\fancyhf{}

\renewcommand{\headrulewidth}{0pt}

% \raggedright
%\DeclareMathOperator*{\E}{\mathbb{E}}


%Neues Kommando Bild
%#1 Größe
%#2 Dateiname
%#3 Bildunterschrift
%#4 Label
\newcommand{\bild}[4]{
	\begin{figure}[htpb]
		\centering
		\includegraphics[width=#1\textwidth]{images/#2}
		\caption{#3}
		\label{#4}
	\end{figure}
}

%Neues Kommando Zitieren
%1 Seite
%2 ID der Quelle
\newcommand{\zitat}[2]{\citep[#1]{#2} }
\pagenumbering{gobble}
\begin{document}
	\setlength{\parindent}{0pt}
	\fancyhf{}
	\fancyhead[L]{\includegraphics[scale=.4]{images/informaticup-logo-with-text.png}}
	\fancyhead[R]{\includegraphics[scale=.1]{images/rwu_logo.png}}
	
\begin{verbatim}
\end{verbatim}

\begin{verbatim}
	
\end{verbatim}
\begin{center}
	\textbf{InformatiCup 2023}
\end{center}
\begin{center}
	\large{\textbf{Profit!}}
\end{center}

\begin{verbatim}
\end{verbatim}
\begin{center}
	von
	\begin{verbatim}
	\end{verbatim}
	\large \textbf{Lisa Binkert, Leopold Gaupe und Yasin Koschinksi}
\end{center}
\begin{center}

	RWU Hochschule Ravensburg-Weingarten University of Applied Sciences
\end{center}

\begin{verbatim}
\end{verbatim}
\begin{center}
	Abgabedatum: 15.01.2023
\end{center}
\begin{verbatim}
	
	
	
\end{verbatim}	

	\renewcommand{\headrulewidth}{0.5pt}
	\setcounter{page}{0}
	\fancyhead[C]{\bfseries\sffamily\thepage}
	\pagenumbering{Roman}
	
	\tableofcontents
	\newpage
	\cleardoublepage
	\addcontentsline{toc}{section}{Abbildungsverzeichnis}
	\listoffigures
	\newpage
	\addcontentsline{toc}{section}{Tabellenverzeichnis}
	\listoftables
	%Einleitung
	
	\section{Einleitung}
Der InformatiCup ist ein Wettbewerb, der von der Gesellschaft für Informatik jährlich veranstaltet wird. Studierende aller Fachrichtungen an Universitäten und Hochschulen in Deutschland, Österreich und der Schweiz dürfen daran teilnehmen.
\\\\
Dieses Dokument beschreibt die Lösung für den InformatiCup 2023 des Teams “Die Schmetterlinge”. Das Team besteht aus Lisa Binkert, Leopold Gaube und Yasin Koschinski, welche alle an der RWU Hochschule Ravensburg-Weingarten University of Applied Science im Studiengang Master Informatik mit dem Profil Künstliche Intelligenz und Autonome Roboter eingeschrieben sind.
\\\\
Im Kapitel \ref{cap:aufgabenbeschriebung} wird die Aufgabenstellung des Informaticup 2023 genauer erläutert. In Kapitel \ref{cap:loesungsansatze} werden verschiedene Lösungsansätze beschrieben, von denen nur ein Teil in die endgültige Lösung eingeflossen ist (Kapitel \ref{cap:umsetzung}). Die Anwendung der Lösung wird in Kapitel \ref{cap:benutzerhandbuch}, dem Benutzerhandbuch, beschrieben. In Kapitel \ref{cap:evaluation} wird die Lösung evaluiert und bewertet, worauf in Kapitel \ref{cap:fazit} ein Fazit folgt.

	\section{Aufgabenbeschreibung}
Die Aufgabe des InformatiCups 2023 ist das Lösen und Optimieren des rundenbasierten Spiels “Profit!”. 
\\
Das Spiel simuliert rundenbasierte Prozesse, in denen durch das Platzieren von verschiedenen Gebäude Ressourcen abgebaut und Produkte erstellt werden können. Das Herstellen von Produkten wird mit Punkten belohnt, wobei das Ziel das Maximieren dieser Punkte ist.
\\
In den folgenden Abschnitten werden die Regeln und der Ablauf des Spiels sowie die Codierung des Spiels im JSON-Format kurz erläutert.

\subsection{Spielregeln}

\subsection{Spielablauf}

\subsection{Darstellung in JSON}
	
	%Haupteil
	\section{Lösungsansätze}

\subsection{Reinforcement Learning}
\subsection{Deep-Q-Learning}
\subsection{Actor-Critic}
\subsection{Mone-Carlo-Search-Tree}
\subsection{Regelbasierter Ansatz}
	\section{Lösungsumsetzung}\label{cap:umsetzung}

\subsection{Programmiersprache und Bibliotheken}
Das Projekt wurde in der Programmiersprache Python umgesetzt. Diese hat einen einfache und übersichtliche Syntax, sowie große Unterstützung von vielen Bibliotheken im Bereich des Maschinellen Lernens.
\\
Neben Numpy und Pandas zur Datenverarbeitung, wird TensorFlow bzw. Keras genutzt, um verschiedene neuronale Netzwerke aufzubauen und zu trainieren. Die OpenAI Gym vereinfacht die Interaktion zwischen dem Reinforcement Learning Agent mit unserer Implementation der "Profit!"-Umgebung.
\\
Für einen einheitlichen Code wurden folgende Code Konventionen eingehalten.
\begin{table}
	\begin{center}
		\begin{tabular}{ | l | l | } 
		 \hline
			Klassennamen & \dq{}UpperCamelCase\dq{}-Variante \\ \hline
			Konstanten &  \dq{}SNAKE\textunderscore{}CASE\textunderscore{}ALL\textunderscore{}CAPS \dq{}-Variante \\ \hline
			Variablen &  \dq{}lower\textunderscore{}snake\textunderscore{}case \dq{}-Variante\\ \hline
			Funktionen & \dq{}lower\textunderscore{}snake\textunderscore{}case \dq{}-Variante \\ \hline
		\end{tabular}
		\caption{Beschreibung der Code Konventionen }
	\end{center}
\end{table}


\subsection{Grundidee}
Bei “Profit!” handelt es sich um ein Einzelspieler-Spiel mit einer deterministischen Umgebung, in der es keine versteckten Informationen gibt. Eine Implementierung des Spiels wurde über die Webseite “https://profit.phinau.de/” bereitgestellt, in der ein Mensch mittels “drag \& drop” Gebäude platzieren und den Ressourcenabbau simulieren kann. Wir haben diese Umgebung in Python nach implementiert, um die volle Kontrolle über die Spieldynamik zu erhalten. Somit können Reinforcement Learning Agenten uneingeschränkt ihre Umgebung abfragen.
\\\\
Das Spiel hat in seiner Grundform aufgrund des bis zu 100x100 großen Spielfeldes einen sehr großen Zustands- und Aktionsraum, weshalb wir uns dafür entschieden haben, das Problem nach dem Teile-und-Herrsche-Prinzip in mehrere Teilprobleme zu zerlegen.
\\\\
Um eine geeignete Abstraktion zu finden, haben wir uns daran orientiert wie ein Mensch das “Profit”-Spiel spielen würde: Anstatt im ganzen Spielfeld unzusammenhängende Gebäude zu platzieren, geht ein menschlicher Spieler (in der Regel) systematischer vor, indem er als erstes eine Fabrik an einer geeigneten Position platziert und von einer Lagerstätte ausgehend aneinander schließende Gebäude setzt, bis eine Verbindung zur Fabrik hergestellt wurde. Dabei behält ein menschlicher Spieler einen groben Überblick über das ganze Spielfeld, um die ungefähre Richtung, in die er bauen möchte, zu bestimmen. Um mögliche Hindernisse und Sackgassen zu vermeiden oder Förderbänder mit anderen Förderbändern zu überkreuzen, wird die lokale Umgebung des zuletzt gesetzten Gebäudes betrachtet, um das optimale Bauteil auszuwählen. 
\\
Anhand dieser Vorgehensweise ist ersichtlich, dass ein Agent für das Verbinden eines beliebigen Startgebäudes mit einer Fabrik zuständig sein soll. Dieser Agent soll ähnlich wie ein menschlicher Spieler aneinander angrenzende Gebäude platzieren, bis das Zielgebäude  erreicht ist. 
Minen, Förderbänder und Verbinder besitzen alle nur einen Ausgang. Lagerstätten können viele verschiedene Ausgänge haben. Zur vereinfachten Implementierung haben wir uns dafür entschieden, eine Mine als Startgebäude des Agentens anstatt einer Lagerstätte zu verwenden. Somit können wir die Position des Agenten auf den einzigen Ausgang des zuletzt platzierten Gebäudes legen. Des Weiteren müssen nur vier benachbarte Positionen in Betracht gezogen werden, an denen der Eingang eines neuen Gebäudes platziert werden kann. 
\\\\
Um “Profit!” zu lösen, braucht es noch einen weiteren Agenten, der bestimmt, welche Lagerstätte bzw. Mine mit welcher Fabrik verbunden werden soll und was zu tun ist, falls eine Verbindung nicht hergestellt werden kann. Außerdem muss dieser Agent bestimmen, wo Fabriken und Minen platziert werden sollen, damit der erste Agent diese verbinden kann.
\\\\
Im Weiteren wird der erste Agent als untergeordnet und der zweite als übergeordnet bezeichnet. Beide können sowohl durch Reinforcement Learning Methoden trainiert, als auch mit Hilfe eines regelbasierten Ansatzes gelöst werden. 

\subsection{“Profit!” Umgebung}
Der erste Schritt der Lösungsumsetzung war das Nachimplementieren von Profit selber.  Hierfür wurde eine Umgebungsklasse (Environment) erstellt, die wie die Umgebung der Website agieren soll. Die Umgebung kann aus einer JSON-Datei erstellt werden und hat auch die gleichen Eigenschaften. Anfangs beinhaltet sie nur Lagerstätte und Hindernisse. Über Methoden lassen sich dann die in Kapitel \ref{cap:aufgabenbeschreibung} beschriebene Hindernisse hinzufügen.
Eine fehlerhafte Umgebung könnte dafür sorgen, dass die Lösung des Spiels fehlerhaft ist.
Daher war wichtig, dass diese Umgebung mit allen dazugehörigen Klassen korrekt umgesetzt wurde, weshalb Unittests diese getestet haben. 
\\\\
Die folgenden Abschnitte beschreiben, wie sich diese Profit Umgebung zusammensetzt und wie sie getestet wurden.

\subsubsection{Environment-Klasse und Gebäude}
Die Environment-Klasse prüft für jedes neu platzierte Gebäude, ob alle in \ref{cap:spielregeln} definierten Regeln eingehalten werden. Jedes Gebäude speichert alle Referenzen zu an eigenen Ausgängen angrenzenden Gebäuden. Somit ist es einfach rekursiv zu ermitteln, ob eine Lagerstätte mit einer Fabrik verbunden ist.
\\\\
Eine Aufgabe kann entweder aus einer JSON-Datei oder einem übergebenen JSON-String generiert werden. Gelöste Aufgaben werden als JSON-Datei unter /tasks/solutions gespeichert und die Liste an platzierbaren Gebäuden wird ausgegeben. Alternativ kann eine für Menschen lesbare Repräsentation des Spielfeldes auf der Standardausgabe ausgegeben werden.



\subsubsection{Spiel-Simulation}
Die Simulation des Spiels dient zur Überprüfung der erreichten Punktzahl eines gegebenen Spielfelds. Zusätzlich wird die Anzahl der benötigten Runden ermittelt, um die angegebene Punktzahl zu erreichen. Der Simulator befindet sich in der Datei simulator.py und ist als Klasse implementiert.
\\\\
Bei der Initialisierung einer Klasseninstanz wird dem Konstruktor eine Instanz der Environment-Klasse übergeben. Diese enthält alle Informationen über die auf dem Spielfeld befindlichen Objekte, deren Position und die jeweiligen Verbindungen zwischen ihnen. Außerdem können der Environment-Klasse die für dieses Spiel möglichen Produkte und das dazugehörige Rezept entnommen werden.
\\\\
Jedes platzierbare Objekt verfügt über einen eigenen Cache und einen internen Ressourcenspeicher, welcher die Anzahl der aktuell gehaltenen Ressourcen des jeweiligen Objekts angibt. Zusätzlich hat jedes Objekt typabhängig eine Funktion, welche zum Beginn (start\textunderscore{}of\textunderscore{}round\textunderscore{}action) und zum Ende (end\textunderscore{}of\textunderscore{}round\textunderscore{}action) einer Runde aufgerufen wird.
\\\\
Die start\textunderscore{}of\textunderscore{}round\textunderscore{}action-Funktion des Förderbandes beispielsweise entnimmt alle im Cache befindlichen Ressourcen und fügt diese dem internen Speicher hinzu. Am Ende der Runde überträgt die  end\textunderscore{}of\textunderscore{}round\textunderscore{}action-Funktion die im Speicher befindlichen Ressourcen in den Cache des am Egress befindlichen Objekts.
\\\\
Während der Simulation wird zum Start einer Runde in zufälliger Reihenfolge die start\textunderscore{}of\textunderscore{}round\textunderscore{}action-Funktion jedes Objekts aufgerufen. Anschließend werden die end\textunderscore{}of\textunderscore{}round\textunderscore{}action-Funktionen aufgerufen. Eventuell entstehende Produkte und damit Punkte werden mit der bisherigen Gesamtpunktzahl summiert.
\\\\
Der Simulator gibt nach Abschluss der Simulation die erreichte Gesamtpunktzahl sowie die dafür benötigten Runden zurück.

\subsubsection{Optimale Score}
Mit dem Optimal Score kann für eine gegebene Umgebung ermittelt werden, welche Punktezahl in der Theorie maximal erreicht werden kann. Außerdem wird ermittelt, welche Kombination von Produkten zu dieser optimalen Punktzahl führen. Wenn mehrere Produkte sich Ressourcen teilen müssen, kann es sein, dass es besser ist, ein Produkt wegzulassen, damit dieses einem anderen Produkt nicht die Ressourcen wegnimmt. 
Für die Berechnung wird angenommen, dass ideale Bedingungen herrschen. Die Positionen, an welchen sich die Lagerstätte und Hindernisse befinden, werden nicht in Betracht gezogen. 
Im ersten Schritt wird eine Liste erstellt, die alle möglichen Kombinationen von Produkten enthält. Anschließend wird für jede Produktkombination der Optimal Score berechnet. Der Score und die Produktkombination werden in eine Liste geschrieben, welche sortiert ausgegeben wird.  Die Rückgabe des Optimal Scores ist also eine sortierte Liste an Scores und Produktkombinationen, wobei die beste Kombination an erster Stelle steht.
\\\\
Um den Score für ein einzelnes Produkt zu berechnen, werden die Punktezahl sowie die benötigten Ressourcen des Produkts gebraucht, ebenso wie die Lagerstätte und die Gesamtressourcen der Umgebung. Die Anzahl der Runden wird um zwei reduziert, da mindestens zwei Runden gebraucht werden, bis eine Ressource eine Lagerstätte erreicht.
\\\\
Die Berechnung des Optimal Scores erstellt einen Vektor mit den Ressourcen aller Lagerstätten, die vom Produkt gebraucht werden. Ebenso wird ein Vektor erstellt, welcher die maximale Anzahl aller Minen beinhaltet. Eine Lagerstätte von der Breite b und der Höhe h kann platzbedingt maximal b*h Minen versorgen. Dieser Vektor wird anschließend mit drei multipliziert, da jede Mine pro Runde drei Ressourcen aufnehmen kann.
\\\\
Der Ressourcenvektor wird anschließend durch den Minenvektor geteilt. Dadurch wird ermittelt, wie viele Runden mindestens gebraucht werden, um alle Ressourcen abzubauen. Ist diese Zahl kleiner als die gegebene Rundenzahl-2, dann wird der Ressourcenvektor durch den Produktressourcenvektor geteilt, wodurch sich die Anzahl an maximal zu produzierenden Produkten ergibt. Diese Zahl wird mit der Produktpunktzahl multipliziert, wodurch sich der optimale Score für dieses Produkt ergibt.
Ist die errechnete Mindestrundenzahl größer als die gegebene, so wird ermittelt, wie viele Produkte in den gegebenen Runden maximal produziert werden können. Hier wird der Minenvektor mit der Anzahl der Runden multipliziert und durch den Produktressourcenvektor geteilt. Das Ergebnis sind die maximal reproduzierbaren Produkte, welche mit der Produktpunktzahl multipliziert den optimalen Score ergibt.
\\\\
Wenn mehrere Produkte existieren, so gibt es zwei Möglichkeiten. Entweder die Ressourcen für die Produkte sind unabhängig voneinander, dann wird der Optimal Score aller Produkte addiert, oder Produkte teilen sich Ressourcen. Ist das der Fall, dann wird der Eintrag im Ressourcenvektor durch die Anzahl an Produkten, die sich diese Ressource teilen, geteilt.
Anschließend wird der Optimal Score für jedes Produkt berechnet und addiert.
\\\\
Der Optimal Score entspricht nicht dem tatsächlich möglich erreichbaren besten Wert. Da die Position der Lagerstätten und Hindernisse nicht beachtet wird, kann nicht überprüft werden, wie viele Minen tatsächlich verwendet werden können und wie viele Runden eine Ressource tatsächlich benötigt, um eine Fabrik zu erreichen.
Der Optimal Score soll nur eine Aussage darüber geben, welche Punktzahl für ein Produkt gut ist und wann eine Lösung nicht weiter verbessert werden kann. 

\subsubsection{Testing}
Der Code wurde mit dem Unit Test Framework unittest von Python getestet. Dieses bietet eine einfache Anwendung Testfälle zu definieren und auszuführen. 
Jede Testklasse erbt von unittest.TestCase, die Testfälle werden als Methoden die mit test beginnen angegeben. Über die assert-Funktion können verschiedene Bedingen abgeprüft werden.\zitat{}{unittest}
Die Tests und die Implementierung des Codes wurden parallel ausgeführt von unterschiedlichen Entwicklern. Damit soll sichergestellt werden, dass beispielsweise Denkfehler oder Bugs in der Implementierung weniger wahrscheinlich auch in den Tests vorkommen, wodurch Fehler besser entdeckt werden können. 
Als Hilfe für die Testimplementierung wurde die Webseite \url{https://profit.phinau.de/}  verwendet. Die Webseite wurde für den Informaticup zur Verfügung gestellt und beinhaltet eine interaktive Implementierung des Spiels Profit. Die Implementierung des Spiels sollte identisch zu dem der Webseite sein.
\\\\
\textbf{Environment \& Gebäudeplazierung}\\
In den Environment-Tests werden die verschiedenen Bedingungen der Profit!-Umgebung und ihre Gebäude überprüft.
\\
Es werden mehrere Test-Umgebungen aus einem JSON-String geladen. Die daraus resultierenden Environment-Objekte werden anschließend auf mögliche Fehler in ihrer Darstellung überprüft.
\\
Es gibt viele verschiedene Regeln, welche Gebäude neben welchen gebaut werden dürfen. Die Gebäude-Tests bauen Gebäude auf legale und illegale Weise in eine Umgebung und prüfen, ob diese das neue Gebäude korrekt akzeptiert bzw. verwirft. 
Beispielsweise darf ein Raster nicht von zwei verschiedenen Gebäuden belegt werden oder Förderband-Eingänge dürfen nicht neben Lagerstätten-Ausgängen liegen. Auch die Ausrichtung der Gebäude-Objekte wird überprüft.
In den Tests wird versucht, alle legalen und illegalen Handlungen abzudecken, um sicherzustellen, dass später keine illegalen Aktionen gelernt werden. 
Außerdem wurden Funktionieren zum Platzieren von Gebäuden
\\\\
\textbf{Spielsimulation}\\
Für die Spielsimulation-Tests wird getestet, ob das endgültige Ergebnis des Spiels demselben entspricht wie dem der zur Verfügung gestellten Webseite. Beide Spielsimulationen müssen sowohl die gleiche Punktzahl haben als auch die gleiche Anzahl an Runden. 
\\\\
\textbf{Optimal Score}\\
Der Optimal Score wurde anhand der gegebenen Aufgaben überprüft. Dabei wurde für jede dieser vier Ausgaben von Hand ausgerechnet, was bestenfalls rauskommen kann, was anschließend mit dem Ergebnis des Algorithmus abgeglichen wurde.



\subsection{Untergeordneter Agent}
Für das Verbinden einer Mine mit einer Fabrik haben wir uns für einen Deep Reinforcement Learning Ansatz entschieden. Dabei haben wir viel mit der Netzwerkarchitektur und möglichen Inputs experimentiert, um zwischen Modellkomplexität und benötigter Rechenzeit abzuwägen. 
\\\\
Das Actor-Critic Modell hat im direkten Vergleich zu einem Deep-Q-Netzwerk schlechtere Ergebnisse geliefert und unsere Monte-Carlo-Tree-Search Implementierung hat aufgrund höherer Rechenzeit für das Trainieren keinen Mehrwert geliefert. Nach mehreren erfolglosen Versuchen, die zugrundeliegenden Probleme zu lösen, mussten diese Lösungsansätze aufgegeben werden.
\\\\
Das beste Ergebnis haben wir mit einem Deep-Q-Netzwerk mit zwei versteckten Schichten und insgesamt 215.968 Parametern erzielt. Es bekommt nicht den Zustand des ganzen Spielfeldes, sondern nur einen Ausschnitt davon, sowie vorgefertigte Features als Input.
\subsubsection{Inputs}
Damit unser Agent mit unterschiedlichen Spielfeldgrößen umgehen kann und um die Modellkomplexität in Grenzen zu halten, beschränken wir das Sichtfeld auf einen lokalen 15x15 Bereich, der die Position des Agenten umgibt. 
Für den Agenten ist es wichtig zu wissen, wo sich im Spielfeld nicht-freie Felder, inerten Förderband-Felder, sowie Fabrikeingänge (und Eingänge, die bereits mit der Ziel-Fabrik verbunden sind) befinden. Diese Informationen werden jeweils in den Kanälen eines  15x15x3 binären Tensors codiert. 
Für den Fall, dass das Sichtfeld über den Spielfeldrand hinausragt, wird das Spielfeld durch Hindernisse erweitert.
\\\\
Falls sich kein Fabrikeingang im Sichtfeld des Agenten befindet, sollte dieser zumindest die ungefähre Richtung, in der sich die Ziel-Fabrik befindet, wissen. Deshalb gibt es sechs zusätzliche binäre Inputs, die die relative x- und y-Position (für jeweils niedriger, höher, gleich) angeben.
\\\\
Aus einem uns nicht ersichtlichen Grund versucht der Agent selbst nach längerem Training gelegentlich illegale Aktionen zu tätigen. Dem können wir nur entgegenwirken, indem wir die legalen Aktionen in den Zustand als binärer Input aufnehmen. 
\subsubsection{Outputs}
Wir haben festgestellt, dass Verbinder nur in seltenen Fällen einen Mehrwert gegenüber Förderbändern bieten, da letztere auch zum Kombinieren von mehreren Ressourceflüssen genutzt werden können und gleichzeitig weniger Platz verbrauchen. 
Um den Aktionsraum noch weiter zu reduzieren und um das Training zu beschleunigen, wurden aus diesen Gründen alle Verbinder aus dem Spiel entfernt. Auch Minen sollten nur zum Abbauen von Ressourcen an Lagerstätten benutzt werden und sind somit für den untergeordneten Agenten nicht von Relevanz. Der Agent muss nur noch lernen, Förderbänder an geeigneten umliegenden Positionen zu platzieren. Der Aktionsraum ist somit auf $8 * 4 = 32$ reduziert.

\subsection{Übergeordnete Agent}
\subsection{Wartbarkeit}

	\section{Benutzerhandbuch}\label{cap:benutzerhandbuch}
Dieses Kapitel soll erklären wie mit dem Projekt weitergearbeitet werden kann und wie Änderungen, gerade am Modell des untergeordneten Agenten vorgenommen werden können. Ebenso wird erläutert wie der Code in dem bereitgestellten Docker-Container auszuführen ist.
\\\\
\textbf{Einstellungen}\\
Alle Hyperparameter der Agenten, Einstellungen für den Task Manager, sowie weitere Debug Informationen lassen sich gesammelt in der \textit{settings.py} Datei einstellen. 
Es kann beispielsweise eine andere Modell-Architektur für den untergeordneten Agenten gewählt oder die Hinderniswahrscheinlichkeit des Task Generators angepasst werden. Auch das zuvor erwähnte vereinfachte Spiel kann durch Setzen von \textit{\mbox{SIMPLE\textunderscore{}GAME = True}}
aktiviert werden. 
\\\\
\textbf{Trainieren eines untergeordneten Agenten}\\
Im Anschluss an eine Änderung in \textit{settings.py} muss ein neuer untergeordneter Agent mittels \textit{train\textunderscore{}model.py} trainiert werden. Dies kann je nach Einstellungen und verwendeter Hardware mehrere Stunden in Anspruch nehmen. Um die Trainingszeit zu verringern, sollte in Betracht gezogen werden, die maximale Episodenzahl in \textit{\mbox{settings.py}} zu reduzieren.
\\\\
\textbf{Evaluation von untergeordneten Agenten}\\
Nachdem mehrere Agenten trainiert wurden, kann mithilfe von \textit{evaluate\textunderscore{}model.py} überprüft werden, welches Modell die meisten Aufgaben des Task Generators lösen kann und sich somit am besten in einem Hindernis-Labyrinth bewegt. Standardmäßig wird auf verschiedenen Spielfeldgrößen von $20\times20$, $30\times30$ und $50\times50$ evaluiert.
Das Ergebnis dient nur als relativer Anhaltspunkt. Es sagt nichts darüber aus, wie gut der Agent mit “echten” Aufgaben zurechtkommt, da die Aufgaben des Task Generators sich von Menschen erstellten Aufgaben unterscheiden. 
\\\\
\textbf{Lösen von Aufgaben}\\
Wird \textit{solve\textunderscore{}game.py} mit einem Pfad zu einer JSON-Datei aufgerufen, wird nur diese Aufgabe gelöst. Alternativ kann mit “solve” als Parameter auch eine Aufgabe über die Standardeingabe eingelesen werden. Es werden alle Ausgaben auf die Standardausgabe unterdrückt, bis zum Schluss eine Lösung als Liste von platzierbaren Gebäuden zurückgegeben wird. 
\\\\
Das Ausführen von \textit{solve\textunderscore{}game.py} ohne weitere Parameter bewirkt, dass automatisch alle zuvor definierten Aufgaben nacheinander gelöst werden. Hierbei werden auch die initiale Lösung sowie jede Verbesserung davon auf der Standardausgabe ausgegeben. Um sich einen ausführlichen Lösungsweg anzeigen zu lassen, sollte in \textit{settings.py DEBUG=True} gesetzt werden. Damit werden auch fehlgeschlagene Verbindungswege schrittweise angezeigt.
\newpage
\textbf{Unittests}\\
Durch Aufruf von \textit{all\textunderscore{}unit\textunderscore{}test.py} werden alle in 4.3.4 angegebenen Unittests durchlaufen und auf mögliche Fehler hingewiesen.
\\\\
\textbf{Docker}\\
Das Dockerfile nutzt als Baseimage \textit{tensorflow/tensorflow:2.11.0}, welches über die öffentliche Registry \textit{hub.docker.com} bezogen werden kann. Dieses Image bietet bereits die meisten Pakete, die für das Projekt benötigt werden.
\\\\
Die noch nicht vorinstallierten Pakete werden während des Builds nachinstalliert. Die Pakete werden in der Datei \textit{requirements.txt} aufgelistet, welche sich im selben Ordner wie das Dockerfile befinden muss.
\\\\
Als Entrypoint nutzt das Image den Befehl python \textit{solve\textunderscore{}game.py} solve. Dieses Skript erwartet über die Standardeingabe ein JSON-String, versucht das Problem zu lösen und gibt, ebenfalls über die Standardausgabe, die Lösung als JSON aus.
\\
Um das Image erfolgreich zu bauen, muss folgende Ordnerstruktur vorliegen:
\begin{itemize}
	\item Dockerfile
	\item requirements.txt
	\item informaticup-profit
	\begin{itemize}
		\item solve\textunderscore{}game.py
		\item ...
	\end{itemize}
\end{itemize}

Mit 
\begin{verbatim}docker build . --tag <tag_name>\end{verbatim}
lässt sich ein Build triggern. 
\\
Um einen Container zu starten, wird folgender Befehl genutzt:
\begin{verbatim}docker run -i --rm --network none --cpus 2.000 --memory 2G
	 --memory-swap 2g <tag_name>\end{verbatim}

Im Anschluss erwartet die Standardeingabe eine Aufgaben im JSON-Format. Das Programm errechnet eine Lösung und gibt diese bei der Standardausgabe als Liste von platzierbaren Objekten zurück. Dies kann mit einem großen Spielfeld mehrere Minuten dauern.
\\
Alternativ lässt sich eine Eingabe auch direkt an den Container geben, indem Pipes verwendet werden.
\begin{verbatim}echo ‘{"width":40,"height":20,"objects": ... }’ | docker run 
	-i --rm --network none --cpus 2.000 --memory 2G --memory-swap 2g <tag_name>\end{verbatim}





	
	
	%Schluss
	\section{Evaluation und Diskussion}\label{cap:evaluation}
Grundsätzlich lässt sich sagen, dass das Ergebnis des Projekts positiv zu bewerten ist. Ein Großteil der Aufgaben wird gelöst, einige auch mit sehr hoher Punktzahl und geringer Rundenzahl.
\\\\
Zum Evaluieren des Programms wurden 23 Tasks verwendet. Vier davon sind vom InformatiCup 2023 zur Verfügung gestellt worden, der Rest wurde selbst erstellt. Diese wurden in zwei Kategorien aufgeteilt: einfach und schwer. Die einfachen Aufgaben hatten nur ein einzelnes Produkt definiert, die schweren mehrere. Eine Tabelle mit den Ergebnissen befindet sich im Anhang in Kapitel \ref{cap:statistik}. Es wird dabei bewertet, wie nah das Programm an den optimalen Score herangekommen ist. Hierbei muss jedoch festgehalten werden, dass es nicht immer möglich ist, die optimale Punktzahl zu erreichen. Der Optimal Score soll nur einen Hinweis liefern, wie die Punktzahl interpretiert werden kann. 

\subsection*{Negative Beispiele}
Von den 23 Aufgaben konnte das Programm bei 21 mindestens ein Produkt produzieren und somit Punkte erzielen. Die beiden nicht lösbaren Aufgaben sind die Aufgaben 8 und 18.\\\\
Die Aufgabe 8 besteht aus einer Art Zick-Zack-Weg, durch den Förderbänder gelegt werden müssen. Der problematische Teil ist in Abbildung \ref{fig:not_solvable_task8} mit einem roten Rahmen markiert. Das Programm versucht so schnell wie möglich nach rechts (in Richtung der Fabrik) zu gehen, wodurch es in den Aussparungen der Hindernisse läuft. Dieser falsche Schritt kann dann nicht mehr rückgängig gemacht werden. Die Abbildung \ref{fig:problematic_step} soll zeigen, wie dieser falsche Schritt zur nicht Lösung des Problems führen konnte.
Da das Programm versucht, immer möglichst direkt in die Richtung der Fabrik zu laufen, wurde diese Konstruktion in der Aufgabe dem Programm zum Verhängnis.
\begin{verbatim}
	
	
\end{verbatim}


\bild{1}{not_solvable_task1.jpg}{Nicht lösbare Aufgabe 8}{fig:not_solvable_task8}
\bild{.2}{problematic_step.jpg}{Problematisch Teil der Aufgabe 8}{fig:problematic_step}
\newpage
Die zweite nicht lösbare Aufgabe mit der Nummer 18, welche in Abbildung \ref{fig:not_solvable_task18} dargestellt wird, kann nur durch eine sehr spezifische Gebäude-Kombination gelöst werden. Die Fabrik kann nur an einer Position gesetzt werden, ebenso wie die Minen zu den Lagerstätten. Während die Verbindung von Lagerstätte 0 zur Fabrik mehrere Förderband Kombinationen zulässt, muss die Verbindung von Lagerstätte 1 zur Fabrik mit einem Förderband des Subtypen 4 erstellt werden. Wird vom Agenten fälschlicherweise der Subtyp 1 gewählt, was eine legale Aktion ist, so ist es nicht mehr möglich, die Fabrik mit Förderbänder zu erreichen. \\
Hier ist das Hauptproblem, dass die zwei untergeordneten Agenten nicht miteinander kommunizieren. Der erste Agent erreicht sein Ziel, ohne dabei zu beachten, dass dabei der Weg für den zweiten Agenten versperrt wird. Das gemeinsame Ziel, Produkte zu produzieren, schlägt fehl. 
\bild{.9}{not_solvable_task2.jpg}{Nicht lösbare Aufgabe 18}{fig:not_solvable_task18}
\\\\
In einzelnen Aufgaben lässt sich außerdem eine suboptimale-Strategie des untergeordneten Agenten feststellen. 
Wenn sich die Fabrik direkt hinter einem Hindernis befindet, weiß der untergeordnete Agent nicht in welche Richtung er laufen soll. Dies führt dazu, dass eine nicht-optimale Schleife mit überkreuzten Förderbändern entstehen kann (vgl. Abb. \ref{fig:conveyor-loop}, entnommen aus Lösung der Aufgabe 15). Der Agenten kann lokal gute Entscheidungen treffen, ihm fehlt jedoch der große Überblick über die Gesamtaufgabe.
\bild{.7}{conveyor-loop.png}{Darstellung einer Förderband-Schleife}{fig:conveyor-loop}

\subsection*{Positive Beispiele}
In den 21 gelösten Aufgaben konnte 11 mal der Optimal Score tatsächlich erreicht werden. Besonders hervorzuheben ist die erste der vorgegebenen Aufgaben.  Hier konnte das Programm nicht nur den Optimal Score erreichen, sondern auch eine geringere Rundenanzahl erzielen, als die von den Teammitgliedern erstellte manuelle Lösung. Das Programm erreichte 410 Punkte mit 26 Runden, während die manuelle Lösung 410 bei 43 Runden erzielte. Wenn das Programm eine Lösung gefunden hat, versucht es diese Lösung weiter zu optimieren, um die Punktzahl zu erhöhen und die Rundenzahl zu verringern. Es wird versucht mehrere Minen zu platzieren und diese mit bereits bestehenden Förderbandverbindungen zu verbinden. Dadurch konnte das Programm eine sehr gute Leistung bei dieser Aufgabe erzielen.
Die gelöste Aufgabe wird in der Abbildung \ref{fig:task1_solved} dargestellt.
\bild{.7}{solved_task01.jpg}{Sehr gut gelöste Aufgabe 1}{fig:task1_solved}
\\\\
Des Weiteren soll die Aufgabe 22 hervorgehoben werden.  Diese hat ebenso die bestmögliche Punktzahl erreicht. Die Aufgabe wird in Abbildung \ref{fig:task22_solved} dargestellt. Die beste Lösung für diese Aufgabe zu finden, ist aus menschlicher Sicht nicht intuitiv. Es gibt viele Produkte, die die gleichen Ressourcen benötigen. Der naive Ansatz, alle gegebenen Produkte zu produzieren, ist bei dieser Aufgabe nicht der beste Weg. Die einzelnen Produkte müssen sich Ressourcen teilen, daher ist die beste Variante nur das Produkt 3 zu produzieren, da mit diesem Produkt die meisten Punkte erzielt werden können. Der Optimal Score berechnet neben der besten Punktzahl auch die beste Produktkombination aus. Mit diesem Wissen kann der Agent bei vielen Produkten die beste Produktkombination auswählen, was bei Aufgabe 22 dazu führt, dass nicht einmal versucht wird, die Produkte 0-2 herzustellen. Das Produzieren von Produkt 3 führt zur besten Punktzahl. Die Aufgabe 22 wurde als optionale Beispielaufgabe abgegeben.
\bild{1}{solved_task22.jpg}{Sehr gut gelöste Aufgabe 22}{fig:task22_solved}

\newpage
In 17 Aufgaben wurde eine Punktezahl erreicht, die 50\% oder höher als die jeweiligen Optimal Scores waren. Nur in 6 Fällen lag die erspielte Punktezahl darunter. Durchschnittlich wurden 70\% des Optimalen Scores erreicht, was als gutes Ergebnis bewertet werden kann.

\subsection*{Rechenzeit}
Der Zeitaspekt in der Lösung wird zwar beachtet, aber nicht aktiv bei den erstellten Lösungen überprüft. Der Grund dafür ist, dass die Implementierung, die die Zeit berücksichtigt, erst spät hinzugefügt wurde. Beim Erstellen der Tasks wurde daher auch diesem Punkt wenig Beachtung geschenkt. Weshalb ein Standardwert von 300 Sekunden in allen Tasks verwendet wurde. Dieser Wert konnte von 22 der 23 Aufgaben eingehalten werden, nur die Aufgabe 21 hat diesen Wert mit 395,25 Sekunden überschritten. 
Aufgabe 21 ist ein Spielfeld mit Größe $100 \times 100$, was dem größtmöglichen Spielfeld entspricht. Außerdem umfasst diese Aufgabe große Lagerstätten, wodurch es viele möglich platzierbare Minen gibt. Allgemein kann beobachtet werden, dass größere Spielfelder auch mehr Zeit zum Lösen brauchen. 
\\\\
Durchschnittlich hat das Programm für alle Aufgaben 31 Sekunden. Wenn die Aufgabe 21 in dieser Rechnung ausgeschlossen wird, reduziert sich dieser Wert auf 14,4 Sekunden.













	\section{Fazit}\label{cap:fazit}
	%Literaturverzeichnis
	\addcontentsline{toc}{section}{Literaturverzeichnis}
	\renewcommand{\refname}{Literaturverzeichnis}
	\bibliographystyle{alphadin}
	\bibliography{Quellen}
	
	\section{Anhang}

\subsection{Ergebnisse der Evaluierung}

\begin{table}
	\begin{center}
		\begin{tabular}{ | l | l | l | l | l | l |} 
\hline
\textbf{Aufgabe}&	\textbf{Optimaler Score}&	\textbf{Erreichter Score}&	\textbf{Runden}&	\textbf{Zeit in s}& \textbf{Score in \%}\\\hline		
001.task.json&	410&	410&	26&	7,49&		100\\\hline
002.task.json&	120&	90&	20&	0,1&		75\\\hline
003.task.json&	720&	120&	47&	41,04&		16,67\\\hline
005.task.json&	2220&	1050&	50&	30,06&		47,3\\\hline
006.task.json&	40&	40&	8&	0,23&		100\\\hline
007.task.json&	200&	80&	29&	0,61&		40\\\hline
008.task.json&	520&	0&	0&	2,86&		0\\\hline
009.task.json&	490&	490&	46&	21,13&		100\\\hline
010.task.json&	180&	180&	20&	0,3	&	100\\\hline
011.task.json&	280&	280&	31&	0,82&		100\\\hline
012.task.json&	620&	620&	50&	14,65&		100\\\hline
013.task.json&	300&	300&	44&	2,05&		100\\\hline
014.task.json&	400&	280&	100&	0,4&		70\\\hline
015.task.json&	450&	240&	62&	4,48&		53,33\\\hline
016.task.json&	1030&	970&	100&	93,16&		94,17\\\hline
017.task.json&	1030&	780&	100&	24,67&		75,73\\\hline
018.task.json&	170&	0&	0&	0,27&		0\\\hline
019.task.json&	40&	40&	16&	0,71&		100\\\hline
020.task.json&	150&	150&	37&	5,36&		100\\\hline
021.task.json&	8340&	570&	99&	395,25&		6,83\\\hline
022.task.json&	900&	900&	63&	12,74&		100\\\hline
023.task.json&	280&	280&	38&	46,75&		100\\\hline
\end{tabular}
\caption{Ergebnisse der Evaluierung in Tabellenform}
\end{center}
\end{table}


\newpage
\subsection{Nicht lösbare Aufgaben}
\bild{1}{not_solvable_task1.jpg}{Nicht lösbare Aufgabe 8}{}
\bild{1}{not_solvable_task2.jpg}{Nicht lösbare Aufgabe 18}{}
\newpage
\subsection{Sehr gut gelöste Aufgaben}\label{cap:solved_solutions}
\bild{1}{solved_task01.jpg}{}{}
\bild{1}{solved_task22.jpg}{}{fig_task22}






















	
\end{document}