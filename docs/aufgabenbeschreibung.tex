\section{Aufgabenbeschreibung}\label{cap:aufgabenbeschriebung}
Die Aufgabe des InformatiCups 2023 ist das Lösen und Optimieren des Spiels “Profit!”. 
Das Spiel simuliert rundenbasierte Prozesse, in denen durch das Platzieren von verschiedenen Gebäude Ressourcen abgebaut und Produkte erstellt werden können. Das Herstellen von Produkten wird mit Punkten belohnt. Das Ziel des Spiels ist es die Punkte zu maximieren. Dies sollte möglichst effizient, also mit minimaler Rundenanzahl, erreicht werden.
\\\\
In den folgenden Abschnitten werden die Regeln und der Ablauf des Spiels sowie die Codierung des Spiels im JSON-Format kurz erläutert.

\subsection{Spielregeln}\label{cap:spielregeln}
Das Spielfeld besteht aus einem maximal $100\times100$ großen Rasterfeld. Ein Feld ist entweder leer oder durch ein Objekt besetzt. Das linke obere Feld befindet sich an der Stelle (0,0), das rechte untere an (Breite-1, Höhe-1). Die Größe des Spielfeldes ist für jede Aufgabe vorgegeben. Gebäude können Ein- und Ausgänge besitzen, über die Ressourcen an angrenzende Gebäude weitergegeben werden. An einen Ausgang darf maximal ein Eingang horizontal oder vertikal angrenzen. An einen Eingang dürfen jedoch mehrere Ausgänge anliegen, was bewirkt, dass Ressourceflüsse zusammengeführt werden. 
\\
Zu Beginn des Spiels sind bereits Hindernisse und Lagerstätten mit Ressourcen vorhanden.
Ebenso werden die maximale Rundenanzahl und die Produkte, die produziert werden können, festgelegt.
\\\\
Die Abbildung \ref{fig:task1} zeigt ein mögliches Spielfeld. In diesem Beispiel ist das Feld $30\times20$ groß und enthält drei Lagerstätten mit den Ressourcen Subtyp 0, 1 und 2, sowie zwei Hindernisse.
%#1 Größe
%#2 Dateiname
%#3 Bildunterschrift
%#4 Label
\bild{.7}{task1.jpg}{Beispielumgebung eines \dq{}Profit!\dq{}-Spiels}{fig:task1}


\subsubsection*{Lagerstätte}
Die Menge an Ressourcen einer Lagerstätte wird durch die Größe der Lagerstätte festgelegt.  Diese wird  auf das Fünffache der Anzahl an Lagerstättenfelder gesetzt. Ist also eine Lagerstätte $3 \times 3$ groß dann enthält es $3\cdot 3 \cdot 5 = 45$ Ressourcen eines bestimmten Subtyps. Insgesamt gibt es 8 Subtypen (0-7).

\subsubsection*{Hindernisse}
Hindernisse im Spielfeld stellen Felder dar, in denen kein anderes Gebäude gebaut werden kann. Diese sind beliebig groß, haben aber immer eine rechteckige Form. 

\subsubsection*{Produkt}
Für jedes Spiel ist mindestens ein Produkt definiert, maximal acht. Ein Produkt benötigt eine beliebige Kombination der acht Ressourcen. Die benötigte Menge der jeweiligen Ressource ist ebenfalls definiert. Beispielsweise kann zur Herstellung von Produkt 0 dreimal die Ressource 0 und einmal die Ressource 1 benötigt werden. Jedes Produkt gibt eine bestimmte Anzahl an Punkte.

\subsubsection*{Mine}
Um die Ressourcen in den Lagerstätten abzubauen, gibt es Minen. Sie ist $4\times2$ oder $2\times4$ Felder groß und hat vier Subtypen, die die Rotation der Mine bestimmen. Die Abbildung \ref{fig:mine} zeigt die vier verschiedenen Subtypen der Mine.
Jede Mine hat einen Eingang (+) und einen Ausgang (-). Der Eingang muss an einem Ausgang einer Lagerstätte anliegen, um die Ressourcen abzubauen. An dem Ausgang können Eingänge von Förderbändern, Verbindern oder Fabriken anliegen. 
\bild{0.5}{mines.jpg}{Darstellung der vier Minen-Subtypen}{fig:mine}
\subsubsection*{Förderband}
Um Ressourcen von einer Lagerstätte zu einer Fabrik zu befördern, können Förderbänder genutzt werden. Sie stellen zusätzliche Verbindungsstücke zwischen Gebäuden dar.
\\
Förderbänder haben einen Eingang (+) und einen Ausgang (-) und sind entweder 3 oder 4 Felder lang. Die Abbildung \ref{fig:conveyor} zeigt die je in vier Subtypen der beiden Varianten. Somit hat das Förderband insgesamt acht Subtypen.
Im Gegensatz zu allen anderen Objekten dürfen sich Förderbänder auch kreuzen.
\bild{0.5}{conveyor.jpg}{Darstellung der acht Förderband-Subtypen}{fig:conveyor}

\subsubsection*{Verbinder}
Wenn ein Produkt mehrere Ressourcen benötigt, können diese mit einem Verbinder zusammengeführt und gemeinsam zur Fabrik befördert werden. 
\\
Ein Verbinder hat drei Eingänge (+) und einen Ausgang (-). Auch hier gibt es vier Subtypen, die jeweils die Rotation des Verbinders bestimmen (Abbildung \ref{fig:combiner}).
\bild{0.5}{combiner.jpg}{Darstellung der vier Verbinder-Subtypen}{fig:combiner}

\subsubsection*{Fabrik}
Für jede Fabrik ist definiert, wie viele und welche Ressourcen gebraucht werden, um ein Produkt herzustellen. Eine Fabrik stellt immer nur ein Produkt her. Da es maximal acht Produkte geben kann, gibt es von der Fabrik insgesamt acht verschiedene Subtypen, für jedes Produkt einen. Jede Fabrik ist $5\times5$ groß. Die äußeren Felder sind Eingänge für Ressourcen, insgesamt 16.
\\
Mit einer Kombination aus Mine, Förderband und Verbinder werden die Ressourcen von den Lagerstätten zur Fabrik befördert. Die Abbildung \ref{fig:factory} zeigt, wie so ein Aufbau aussehen kann. Im Beispiel benötigt das Produkt 0 die Ressourcen 0 und 1.
\bild{0.7}{solved_task.jpg}{Lösungsbeispiel einer Aufgabe}{fig:factory}
\newpage
\subsection{Spielablauf}
Das Spiel läuft rundenbasiert ab. Jede Runde beginnt mit einer “Beginn der Runde”-Aktion und endet mit einer “Ende der Runde”-Aktion.
Die Tabelle \ref{tab:object_actions} definiert die Aktionen der einzelnen Objekte. 
\begin{table}[htp]
\begin{center}
\begin{tabular}{ | p{2cm} | p{5cm} | p{6,5cm}| } 
	\hline
	\textbf{Objekt}& \textbf{Beginn der Runde} & \textbf{Ende der Runde} \\ \hline
	Mine & Nimmt Ressourcen auf & Gibt angenommene Ressourcen weiter \\ \hline
	Förderband & Nimmt Ressourcen auf & Gibt angenommene Ressourcen weiter \\ \hline
	Verbinder & Nimmt Ressourcen auf & Gibt angenommene Ressourcen weiter \\ \hline
	Fabrik & Nimmt Ressourcen auf & Produziert so viele Produkte wie Ressourcen verfügbar sind \\ \hline
	Lagerstätte & - & Gibt bis zu 3 Ressourcen an jeden Eingang einer Mine \\ \hline
\end{tabular}
\caption{Beschreibung der Aktionen der einzelnen Objekte}\label{tab:object_actions}
\end{center}
\end{table}
Das Spielfeld hat neben der Feldgröße und den vorhandenen Objekten auch das Attribut “turns”. Dieses Attribut gibt die maximal erlaubte Anzahl an Spielrunden an.
\\\\\
Pro Runde werden Ressourcen von Lagerstätten mit Hilfe von Minen abgebaut und Stück für Stück über die gebauten Förderbänder und gegebenenfalls Verbinder zu einer Fabrik befördert. 
Die Produktion einer Fabrik stoppt, wenn die für das Produkt benötigten Ressourcen vollständig abgebaut und zu dieser Fabrik befördert sind.
Ist das bei allen gebauten Fabriken der Fall, so endet das Spiel und der Spieler erhält eine Gesamtpunktzahl und die Anzahl der dafür benötigten Runden. Das Spiel kann auch vorzeitig beendet werden, wenn die vordefinierte maximale Rundenanzahl erreicht wird.
\\\\
Das Ziel des Spiels ist es, die Punktezahl zu maximieren und dabei als Nebenziel die Rundenanzahl zu minimieren.


\subsection{Darstellung in JSON}
Der Input des Spiels erfolgt im JSON-Format (JavaScript Object Notation), in dem Lagerstätten, Hindernisse und Produkte definiert sind. Das JSON für das Beispiel in Abbildung \ref{fig:task1} wird in der Abbildung \ref{fig:task1_json} dargestellt.
\bild{.85}{task1_json.jpg}{JSON Darstellung einer Aufgabe}{fig:task1_json}
Das Spielfeld wird durch die angegebene Breite und Höhe definiert. Es enthält fünf verschiedene Objekte, drei Lagerstätten und zwei Hindernisse. Jedes dieser Objekte besitzt eine x- und y-Koordinate, die die Position im Feld bestimmt, sowie eine Höhe und eine Breite. Lagerstätten haben einen Subtyp, welcher die hier verfügbare Ressource festlegt. Nach den Objekten werden die Produkte definiert. Im JSON sind hierfür der Subtyp, die benötigten Ressourcen pro Produkt und die Anzahl an Punkten pro Produkt angegeben. Die Position der Ressource im Array bestimmt den jeweiligen Subtyp.
\\\\
Jedes Objekt kann durch einen JSON-String dargestellt werden, der die Eigenschaften des Objekts definiert. In folgender Tabelle \ref{tab:json} wird für jeden Typ ein beispielhafter JSON-String aufgeführt:

\begin{table}[htp]
	\begin{center}
		\begin{tabular}{ | l | l | } 
			\hline
			\textbf{Objekt}& \textbf{JSON}\\  \hline
			Mine & \{"type":"mine", \dq{}subtype":0, "x":0, "y":0\}\\ \hline
			Förderband & \{"type":"conveyor", \dq{}subtype":0, "x":0, "y":0\} \\ \hline
			Verbinder & \{"type":"combiner", \dq{}subtype":0, "x":0, "y":0\} \\ \hline
			Fabrik & \{"type":"factory", \dq{}subtype":0, "x":0, "y":0\}\\ \hline
			Lagerstätte & \{"type":"deposit", \dq{}subtype":0, "x":0, "y":0, "width":1, "height":1\}  \\ \hline
			Hindernisse & \{"type":\dq{}obstacle", "x":0, "y":0, "width":1, "height":1\}\\\hline
		\end{tabular}
		
		\caption{Beschreibung der JSON Struktur einzelner Objekte }\label{tab:json}
	\end{center}
\end{table}



