\section{Fazit}\label{cap:fazit}
Wie in Kapitel \ref{cap:loesungsansatze} beschrieben, gab es sehr viele Ideen, wie die Aufgabe des InformatiCup 2023 gelöst werden könnte. Leider konnte in den drei Monaten, die für diese Aufgabe Zeit war, nicht alles umgesetzt und ausprobiert werden. Zeitbedingt konnten daher Ansätze wie AC oder MCST nicht erfolgreich implementiert werden. Ein Großteil der Zeit wurde für die fehlerfreie Implementierung der  “Profit”-Umgebung  investiert, was zu einem  Zeitdruck am Ende des Projekts führte. Da der untergeordnete Agent auf diese Umgebung trainiert wurde, musste sichergestellt werden, dass diese korrekt implementiert wurde.
\\\\
Der Aufwand, eine ML-Lösung zu entwickeln, wurde am Anfang des Projekts unterschätzt. Es ist daher möglich, dass ein regelbasierter “Pathfinding”-Ansatz für den untergeordneten Agenten weniger zeitintensiv zu entwickeln gewesen wäre.
\\\\
Mit mehr Zeit wäre es möglich gewesen, die Lösungen des Programms noch weiter zu optimieren, so dass Aufgaben, die nicht oder nicht gut gelöst wurden, ebenfalls eine gute Punktzahl erreichen. Der Brute-Force Ansatz, der für die Positionierung der Fabrik gewählt wurde, könnte durch ein gezieltes Platzieren verbessert werden, so dass der untergeordnete Agent kürzere Wege zur Fabrik hat. Auch das Platzieren von mehreren Fabriken könnte dadurch besser gestaltet werden.
\\\\
Wie sich zeigt, gibt es noch Potenzial, die Lösung des Projekts zu verbessern, was nicht bedeutet, dass das Ergebnis des Projekts schlecht ist. Manuell erstellte Lösungen sind zwar in vielen Fällen gleich gut oder sogar besser, trotzdem konnte das Programm menschlich erstellte Lösungen schlagen.
\\\\
Die Evaluation zeigt, dass es deutlich mehr gute und sehr gute Ergebnisse gab als schlechte. Fast die Hälfte der Aufgaben hat das bestmögliche Ergebnis erreicht, was eine sehr gute Statistik ist, wenn beachtet wird, dass der optimale Score nicht immer erreicht werden kann. Daher lässt sich sagen, dass das Resultat dieses Projekt als erfolgreich angesehen werden kann.
\\\\
Das Team “Die Schmetterlinge“ bedankt sich bei der Gesellschaft für Informatik für das Ausrichten des InformatiCups 2023, bei Prof. Markus Schneider für die Betreuung des Projektes und ist  gespannt, welche Lösungen die anderen Teams präsentieren werden.
